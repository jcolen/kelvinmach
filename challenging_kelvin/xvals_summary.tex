\documentclass[letterpaper,11pt]{article}
\usepackage{amsmath}
\usepackage[left=1in, right=1in, bottom=1in, top=1in]{geometry}

\begin{document}

We are aiming to numerically calculate a Fourier integral using the FFT method.

\[f(x) = \int_R F(k) e^{2 \pi i k x} dk \]

\[ f(x) = \sum_{-\infty}^{\infty} F(k) e^{2 \pi i k x} \Delta_ k \]

\[ \approx \sum_{-k_{max}}^{k_{max}} F(k) e^{2 \pi i k x} \Delta_k \]

\[ = \sum_{m = -M}^{M-1} F(m \Delta_k) e^{2 \pi i m \Delta_k x} \Delta_k \text{ where } k_{max} = M \Delta_k \]

\[ = \Delta_k e^{-2 \pi i M \Delta_k x} \sum_{m=0}^{2M-1} F((m - M) \Delta_k) e^{2 \pi i m \Delta_k x} \]

This form looks similar to that of the Inverse Discrete Fourier Transform, shown below:

\[ a_n = \frac{1}{N} \sum_{m = 0}^{M-1} A_k e^{2 \pi i \frac{n m}{M}} \text{ for } n = 0 ... M - 1 \]

Let us discretize $x$ to see further similarities.

\[ \text{Let } x \rightarrow x_n = \Delta_x n \text{ for } n = -M ... M-1 \]

\[ f(x_n) = \Delta_k e^{-2 \pi i k_{max} \Delta_x n} \sum_{m = 0}^{2M-1} F((m - M) \Delta_k) e^{2 \pi i m n \Delta_x \Delta_k} \text{ for } n = -M ... M-1\]

\[ f(x_n) = \Delta_k e^{-2 \pi i k_{max} \Delta_x (n - M)} \sum_{m = 0}^{2M-1} F((m -M) \Delta_k) e^{2 \pi i m n \Delta_x \Delta_k} e^{-2 \pi i m M \Delta_x \Delta_k} \text{ for } n = 0 ... 2M - 1 \]

We can recover the Inverse DFT form if the following condition holds:

\[\Delta_x \Delta_k = \frac{1}{M} \rightarrow \Delta_x = \frac{1}{M \Delta_k} = \frac{1}{k_{max}} \]

\[ \text{Therefore } x_{max} = \Delta_x M = \frac{1}{\Delta_k} = \frac{M}{k_{max}} \]

Since $\Delta_x$ scales with $k_{max}$, the only way to obtain higher resolution is to increase the cutoff point in k-space.

\end{document}
